
Pour comprendre le système de maintenance semi-automatique, il faut expliquer le fonctionnement de prolog.

Prolog fonctionne par système de règles, posées dans des fichiers de code. Il est possible de déclarer de manière générique une règle en posant sa structure (nom, nombre d'arguments, etc. ), puis d'en implémenter plusieurs à partir d'elle ensuite.

Nous avons construit une base de fait  partir d'une règle qui permet de déclarer d'autres règles.

\begin{lstlisting}[frame=single]
reportDefineRule(RulePredicate)
\end{lstlisting}

où les RulePredicate prennent la forme de 

\begin{lstlisting}[frame=single]
reportRuleZ(A,B,X) 
\end{lstlisting}

où Z est le nom de la règle particulière 
\newline
A les cases cochées par le conducteur A
\newline
B les cases cochées par le conducteur B
\newline
et X le tort calculé de A
\newline

Lorsqu'un utilisateur conteste une évaluation des torts, l'interface lui demande d'entrer les nouveaux torts que celui-ci attendait. Le programme créera ensuite une nouvelle règle nommée exceptionY (Y un numéro qui s'incrémente) qui sera testée avant toute les autres. Ainsi pour le cas particulier qui a ammené l'utilisateur à contester le calcul il y aura maintenant une règle propre. 


