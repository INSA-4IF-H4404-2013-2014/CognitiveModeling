\chapter{Application}

\section{Noyau}

Nous avons eu le plssir, l'honneur et l'avantage d'implementer notre application expert a l'aide de SWI-Prolog car ceci est une revolution.
L'avantage de prolog est l'utilisation des predicats pour l'implementation des regles que l'expert doit appliquer. Considerons par exemples :

D'abord, je regarde si un des conducteurs a fait une faute grave. De deux choses l'une : soit la case qui correspond aux cas graves - la case
17 "N'avait pas observé un signal de priorité ou un feu rouge"- a été cochée, soit dans les observations manuscrites il est signalé une
infraction du type : non respect d'un stop, d'un panneau d'interdiction de dépasser ou d'un sens interdit. Dans ce cas, le conducteur a tous
les torts : 100\%.

Nous pouvon en debduir la regles~:

\[A_{17} \Rightarrow A_{torts} = 100\% \cdot B_{torts} = 0\%\]

Alors on en deduit le predicat~:

\begin{lstlisting}[language=Prolog,frame=single]
reportEvaluateFatalMistake(A,_,100) :-
    reportIsChecked(A,c17).
\end{lstlisting}

Seulement, nous avons voulu que le noyeau utilise les regles de maniere deterministe. Alors nous avons un predicat dynamic listant l'ensemble
des regles de l'expert qu'il parcour ensuite a l'evaluation des torts~:

\begin{lstlisting}[language=Prolog,frame=single]
:- dynamic reportEvaluateRules/1.
:- retractall(reportEvaluateRules(_)).

reportDefineRule(RulePredicate) :-
    reportEvaluateRules(RulePredicate) -> (
        true
    ); (
        assert(reportEvaluateRules(RulePredicate))
    ).
\end{lstlisting}

Ainsi il nous faut simplement dir au noyaux de considerer la regle reportEvaluateFatalMistake/3~:

\begin{lstlisting}[language=Prolog,frame=single]
:- reportDefineRule(reportEvaluateFatalMistake).
\end{lstlisting}

Et si les 2 conducteurs font une infraction grave ?
C'est rare, mais dans ce cas, on partage les torts : 50\% chacun.

Nous arions pus coder une regles du genre~:

\begin{lstlisting}[language=Prolog,frame=single]
reportEvaluateFatalMistake50(A,B,50) :-
    reportIsChecked(A,c17),
    reportIsChecked(B,c17).
\end{lstlisting}

Mais sela duplique le code pour chaque regle, pouvant etre la cause d'une erreure. Seulement, le noyeau va tester~:

\begin{lstlisting}[language=Prolog,frame=single]
reportEvaluateFatalMistake(A,B,TortsA).
\end{lstlisting}

Mais aussi~:

\begin{lstlisting}[language=Prolog,frame=single]
reportEvaluateFatalMistake(B,A,TortsB).
\end{lstlisting}

Alors, si les deux predicats sont verifi\'e, le noyaux en d\'eduit automatiquement~:

\[A_{torts} = 50\% \cdot B_{torts} = 50\%\]

\section{Test unitaires}

