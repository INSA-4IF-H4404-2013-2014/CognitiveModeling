\chapter{Application}

\section{Noyau}
hello world

\section{Test unitaires}

A chaque implémentation d'une règle, un test unitaire est codé : il construit un rapport avec certaines cases cochées, et on donne la règle qui doit s'exécuter.

En cas de collision, la procédure qui exécute tous les tests s'arrête, et donne la règle qui a effectivement été utilisée : soit la nouvelle règle est redondante et est supprimée, soit l'une des règles doit être précisée. Généralement, on lui ajoute le prédicat "not(autreRegle)". 

Par exemple, la règle 132 qui donne les torts quand l'accident a eu lieu alors que les deux véhicules roulaient en sens inverse entrait en collision avec la règle 131 qui donne les torts si l'accident se produit parce que l'un des véhicules a franchi l'axe médian, alors que les deux roulaient en sens inverse.
La règle 132, qui avait cette forme :

\begin{center}\textit{reportRule132(A,B,50) :-
    reportReversedWays(A,B).}\end{center} 

Est devenue :

\begin{center}\textit{reportReversedWays(A,B),
    not(reportRule131(A,B,_)).}\end{center} 