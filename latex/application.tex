\chapter{Application}

\section{Noyau}

Nous avons eu l'opportunité d'implémenter notre système expert à l'aide de \texttt{SWI-Prolog}, le langage \texttt{Prolog} étant particulièrement bien adapté à ce type d'application.
L'avantage de \texttt{Prolog} est l'utilisation des prédicats pour l'implémentation des règles que l'expert doit appliquer. Considérons par exemple~:

\textit{D'abord, je regarde si un des conducteurs a fait une faute grave. De deux choses l'une~: soit la case qui correspond aux cas graves ---~la case
17 «~N'avait pas observé un signal de priorité ou un feu rouge~»~--- a été cochée, soit dans les observations manuscrites il est signalé une
infraction du type~: non-respect d'un stop, d'un panneau d'interdiction de dépasser ou d'un sens interdit. Dans ce cas, le conducteur a tous
les torts~: 100\%.}

Nous pouvons déduire la règle~:

\[A_{17} \Rightarrow A_{torts} = 100\% \cdot B_{torts} = 0\%\]

Alors, on en tire le prédicat~:

\begin{lstlisting}[language=Prolog,frame=single]
reportEvaluateFatalMistake(A,_,100) :-
    reportIsChecked(A,c17).
\end{lstlisting}

Néanmoins, nous avons voulu que le noyau utilise les règles de manière déterministe. De fait, nous avons créé un prédicat dynamique listant l'ensemble
des règles de l'expert qu'il parcourt ensuite à l'évaluation des torts~:

\begin{lstlisting}[language=Prolog,frame=single]
:- dynamic reportEvaluateRules/1.
:- retractall(reportEvaluateRules(_)).

reportDefineRule(RulePredicate) :-
    reportEvaluateRules(RulePredicate) -> (
        true
    ); (
        assert(reportEvaluateRules(RulePredicate))
    ).
\end{lstlisting}

Ainsi, il nous faut simplement dire au noyau de considérer la règle \texttt{reportEvaluateFatalMistake/3}~:

\begin{lstlisting}[language=Prolog,frame=single]
:- reportDefineRule(reportEvaluateFatalMistake).
\end{lstlisting}

\textit{Et si les 2 conducteurs commettent une infraction grave~?
C'est rare, mais dans ce cas, on partage les torts~: 50\% chacun.}

Nous aurions pu implémenter une règle de la forme suivante~:

\begin{lstlisting}[language=Prolog,frame=single]
reportEvaluateFatalMistake50(A,B,50) :-
    reportIsChecked(A,c17),
    reportIsChecked(B,c17).
\end{lstlisting}

Mais cela duplique le code pour chaque règle, et peut être la cause d'une erreur. Toutefois, le noyau va tester~:

\begin{lstlisting}[language=Prolog,frame=single]
reportEvaluateFatalMistake(A,B,TortsA).
\end{lstlisting}

Mais aussi~:

\begin{lstlisting}[language=Prolog,frame=single]
reportEvaluateFatalMistake(B,A,TortsB).
\end{lstlisting}

Alors, si les deux prédicats sont vérifiés, le noyau en déduit automatiquement~:

\[A_{torts} = 50\% \cdot B_{torts} = 50\%\]

\section{Tests unitaires}

À chaque implémentation d'une règle, un test unitaire est codé~: il construit un rapport avec certaines cases cochées, et on donne la règle qui doit s'exécuter et le r\'esultat attendu.

En cas de collision, la procédure qui exécute tous les tests s'arrête, et donne la règle qui a effectivement été utilisée~: soit la nouvelle règle est redondante et est supprimée, soit l'une des règles doit être précisée. Généralement, on lui ajoute le prédicat \texttt{not(autreRegle)}.

Par exemple, la règle 132 qui donne les torts quand l'accident a eu lieu alors que les deux véhicules roulaient en sens inverse entrait en collision avec la règle 131 qui donne les torts si l'accident se produit parce que l'un des véhicules a franchi l'axe médian, alors que les deux roulaient en sens inverse.
La règle 132, qui avait cette forme~:

\begin{lstlisting}[language=Prolog,frame=single]
reportRule132(A,B,50) :-
    reportReversedWays(A,B).
\end{lstlisting}

Est devenue~:

\begin{lstlisting}[language=Prolog,frame=single]
reportReversedWays(A,B),
    not(reportRule131(A,B,_)).
\end{lstlisting}

La majorité des tests unitaires implémentés sont les exemples de constats fournis avec le sujet. Nous les avons complétés avec d'autres tests appelant d'autres règles pour vérifier de manière exhaustive leur validité.

